\documentclass{scrartcl}

\usepackage[tmargin=1.5cm, 
bmargin=2.5cm, 
lmargin=2.5cm,
rmargin=1.5cm]{geometry}

\usepackage[utf8]{inputenc}
\usepackage{cmbright}

% Mengen Zeichen
\usepackage{amsmath} 
\usepackage{amssymb}

\usepackage[singlespacing]{setspace} % Zeilenabstand

\usepackage{tabularx}
\usepackage{xcolor,colortbl}

\usepackage{graphicx}
%\setlength{\parindent}{0pt}

\renewcommand{\contentsname}{Inhaltsverzeichnis}

\title{Projekt Chemisches Rauschen}
\author{Anja Seegebrecht, Julia Kirchner}
\date{Wintersemester 2015-2016}

\begin{document}

\maketitle

\newpage

\tableofcontents

\newpage

\section{Einleitung}


\section{Grundlagen}
\subsection{Exponential- und Gammaverteilung}
Bei der Exponentialverteiung handelt es sich um eine stetige Wahrscheinlichkeitsverteilung über dem Intervall $[0, +\infty]$. Häufig kommt sie zur Anwendung, wenn nach der Zeit gesucht wird, die zwischen zwei Ereignissen verstreicht.
Ihre Dichtefunktion lautet
\begin{align*}
    f(t) = \begin{cases} \lambda \cdot e^{-\lambda t} \qquad \text{falls } t \geq 0 \\ 0 \qquad \text{falls } t < 0 \end{cases}, \lambda \in \mathbb{R}^+.
\end{align*}
Hierbei bezeichnet $\lambda$ die Ereignisrate (später die Reaktionsrate). Das heißt, $\lambda$ gibt die durchschnittliche Anzahl der Ereignisse an, die in einer Zeiteinheit stattfinden. Durch Verwendung der Exponentialverteilung erhält man die Wahrscheinlichkeit, dass ein Ereignis zwischen $t_0=0$ und $t$ stattfindet. Ein Anwendungsgebiet dieser Verteilung ist der radioaktive Zerfall von Atomen. %Meistens ist die tatsächliche Funktion keine Exponentialverteilung und wird lediglich als solche approximiert, da man leichter mit ihr arbeiten kann. Bedingung für die Annäherung sind die poissonschen Annahmen.
\subsection{Chemische Kinetik}
Die chemische Kinetik ist ein Teilgebiet der physikalischen Chemie. Sie untersucht den zeitlichen Ablauf chemischer Reaktionen und chemisch-physikalischer Vorgänge z.B. Diffusionen. Die grundlegende Größe der Kinetik ist die Reaktionsgeschwindigkeit $v_R$. Die Geschwindigkeit einer chemischen Reaktion gibt die zeitliche Änderung der Stoffmenge oder der Konzentration an.
Die Reaktionsgeschindigkeit hat stets einen positiven Wert.
So gilt für eine Elementarreaktion  $A \rightarrow B$
\begin{align*}
     v_R = - \frac{\Delta c_A}{\Delta t} = \frac{\Delta c_B}{\Delta t}
\end{align*}
 
 
\subsection{Monte-Carlo-Schritt}

Die Grundlage der Monte-Carlo-Methode ist es, für zufällig ausgewählte Parameter über bestimmte Zusammenhänge die zugehörigen Ergebnisse zu ermitteln.
\subsection{Gillespie Algorithmus}
\begin{itemize}
    \item{Initialisierung}: Festlegung der Anfangsanzahl der Moleküle im System, der Reaktionskonstanten und dem Zufallszahlengenerator.
\item{Monte Carlo Schritt}: Eine Zufallszahl generieren, um die nächste Reaktion festzulegen und das Zeitintervall, nach dem sie eintreten soll. Die Wahrscheinlichkeit mit der eine Reaktion eintritt, ist proportional zu der Anzahl von Substratmolekülen.
\item{Aktualisieren}: Die Zeitzählung um das Zufallszeitintervall aus Schritt zwei erhöhen und die Anzahlen der Moleküle aktualisieren
\item{Iterieren}: Immer wieder von Schritt zwei beginnen, bis die Anzahl der Edukte null ist oder die Simulationszeit überschritten wurde.
\end{itemize}

\section{Radioaktiver Zerfall}
\subsection{Differentialgleichung}
\subsection{Übergangsrate}
\subsection{Simulation}
\subsection{Vergleich und Fazit}
\subsection{*Zusatz: Mittlere Zeit bis zum vollständigen Zerfall}

\section{Brusselatormodell}
\subsection{Differentialgleichung}
\subsection{Übergangsraten}
\subsection{Simulation}
\dots 
\section{Simulation von Ratenübergängen}
\section{Von Reaktionsraten und Übergangsraten}

\end{document}
