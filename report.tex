\documentclass{scrartcl}

\usepackage[tmargin=1.5cm, 
bmargin=2.5cm, 
lmargin=2.5cm,
rmargin=1.5cm]{geometry}

\usepackage[utf8]{inputenc}
\usepackage{cmbright}

% Mengen Zeichen
\usepackage{amsmath} 
\usepackage{amssymb}

\usepackage[singlespacing]{setspace} % Zeilenabstand

\usepackage{tabularx}
\usepackage{xcolor,colortbl}

\usepackage{graphicx}
%\setlength{\parindent}{0pt}

\renewcommand{\contentsname}{Inhaltsverzeichnis}

\title{Projekt Chemisches Rauschen}
\author{Anja Seegebrecht, Julia Kirchner}
\date{Wintersemester 2015-2016}

\begin{document}

\maketitle

\newpage

\tableofcontents

\newpage

\section{Einleitung}


\section{Grundlagen}
\subsection{Exponential- und Gammaverteilung}
Bei der Exponentialverteiung handelt es sich um eine stetige Wahrscheinlichkeitsverteilung über dem Intervall $[0, +\infty]$. Häufig kommt sie zur Anwendung, wenn nach der Zeit gesucht wird, die zwischen zwei Ereignissen verstreicht.
Ihre Dichtefunktion lautet
\begin{align*}
    f(t) = \begin{cases} \lambda \cdot e^{-\lambda t} \qquad \text{falls } t \geq 0 \\ 0 \qquad \text{falls } t < 0 \end{cases}, \lambda \in \mathbb{R}^+.
\end{align*}
Hierbei bezeichnet $\lambda$ die Ereignisrate (später die Reaktionsrate). Das heißt, $\lambda$ gibt die durchschnittliche Anzahl der Ereignisse an, die in einer Zeiteinheit stattfinden. Durch Verwendung der Exponentialverteilung erhält man die Wahrscheinlichkeit, dass ein Ereignis zwischen $t_0=0$ und $t$ stattfindet. Ein Anwendungsgebiet dieser Verteilung ist der radioaktive Zerfall von Atomen. %Meistens ist die tatsächliche Funktion keine Exponentialverteilung und wird lediglich als solche approximiert, da man leichter mit ihr arbeiten kann. Bedingung für die Annäherung sind die poissonschen Annahmen.
\subsection{Chemische Kinetik}
\subsection{Monte-Carlo-Schritt}
\subsection{Gillespie Algorithmus}

\section{Radioaktiver Zerfall}
\subsection{Differentialgleichung}
\subsection{Übergangsrate}
\subsection{Simulation}
\subsection{Vergleich und Fazit}
\subsection{*Zusatz: Mittlere Zeit bis zum vollständigen Zerfall}

\section{Brusselatormodell}
\subsection{Differentialgleichung}
\subsection{Übergangsraten}
\subsection{Simulation}
\dots 
\section{Simulation von Ratenübergängen}
\section{Von Reaktionsraten und Übergangsraten}

\end{document}
